\documentclass[10pt]{article}

%%% Input encoding
\usepackage[utf8]{inputenc} % set input encoding (not needed with XeLaTeX)

%%% Article customization
\usepackage{geometry}
\geometry{a4paper} % letter paper a5paper etc.
\geometry{margin=1in}
% \geometry{landscape}

%%% Indentation
\usepackage{parskip} % begin line with indent
% \usepackage[parskip]{parskip} % begin paragraph with empty line instead of indent
\setlength\parindent{15pt} % set indent length
% \setlength\topskip{10pt} % vertical margin for only first row

% the following can change the line spacing, but use setspace instead.
% \setlength{\parskip}{\baselineskip}
% \setstretch{2.0}
\usepackage{setspace}


%%% Basic packages
\usepackage{graphicx} % support the \includegraphics command and options
\usepackage{booktabs} % for much better looking tables
\usepackage{array} % for better arrays (eg matrices) in maths
\usepackage{paralist} % very flexible & customizable lists (eg. enumerate/itemize, etc.)
\usepackage{verbatim} % adds environment for commenting out blocks of text & for better verbatim
\usepackage{subfig} % make it possible to include more than one captioned figure/table in a single float
\usepackage{amsmath}
\usepackage{amssymb}
\usepackage[ruled]{algorithm2e} % for pseudocode generation. ruled makes the vertical lines.
\usepackage{listings} % for presenting codes
\lstset{
  % language=C,
  basicstyle=\small,
  breaklines=true
  }
\usepackage{kotex}
\usepackage{underscore}
\usepackage{bookmark}
\usepackage{subfiles}

%%% HEADERS & FOOTERS
\usepackage{fancyhdr} % This should be set AFTER setting up the page geometry
\pagestyle{fancy} % options: empty , plain , fancy
\renewcommand{\headrulewidth}{0pt} % customize the layout...
\lhead{}\chead{}\rhead{}
\lfoot{}\cfoot{\thepage}\rfoot{}


%%% SECTION TITLE APPEARANCE
\usepackage{sectsty}
\allsectionsfont{\sffamily\mdseries\upshape} % (See the fntguide.pdf for font help)


%%% ToC (table of contents) APPEARANCE
\usepackage[nottoc,notlof,notlot]{tocbibind} % Put the bibliography in the ToC
\usepackage[titles,subfigure]{tocloft} % Alter the style of the Table of Contents
\renewcommand{\cftsecfont}{\rmfamily\mdseries\upshape}
\renewcommand{\cftsecpagefont}{\rmfamily\mdseries\upshape} % No bold!


\usepackage{ragged2e}
\usepackage[english]{babel}
\usepackage{blindtext}
\usepackage{hyperref}
\usepackage{titling}
\usepackage{caption}


\title{Manual for Debian Init}
\author{KangSK}
\date{October 28, 2020}


\begin{document}
\doublespacing

\maketitle\label{Title}

\section*{Introduction}

	\textbf{Shortcuts} are one of the best friends for a programmer, a writer, or basically anyone who works with a computer.
	However, multiple environments have their own unique keybindings, making users hard to distinguish between them.
	Of course, a user may fix his/her working environment, especially the working OS, to further fullfil their needs.
	However, this may not be possible, or even if possible, be a very tiresome job for some people, including myself.

	Before going any further, I admit I used the Windows OS for quite a long time, and I still need to go back and forth from Linux to Windows.
	Additionally, I admit, therefore I am more comfortable with the shortcuts on the Windows OS.
	Although yes, I can get used to multiple sets of shortcuts, but I don't really think this is necessary, if I can simply modify the settings instead.
	This was the motivation for making this repository in the first place, and here it is, the manual for it.

	Some might argue that this ia a ``sacrilegious'' act, ignoring the design choices of the original bindings.
	Some argue that I am limiting myself from being able to use any other computer environment that my own.
	I agree with some parts of the idea, but I still will carry on.

	Therefore, I state there the objective of this manual.
	First of all, this manual and all the shortcuts in it is based on \textbf{my} perspective, and is aiming for \textbf{my} convenience.
	Modifying shortcuts of the Windows OS is much harder compared to changing settings on a GNU/Linux environment, therefore this manual is closer to the Windows' shortcuts.
	This will result a Debian environment as convenient as possible for pre-Windows users.

	\newpage

\section{Debian Shortcuts}
\section{Terminal Shortcuts}
\section{VS Code Shortcuts}

\end{document}